% Diagramação do ToC

\setupindenting[yes,next,4ex]     % Indentação


\setupheadtext[content=Sumário]

\def\DotAfterNumber#1{\doiftext{#1}{Capítulo #1}}

\setuphead  [chapter]    [number=yes,numbercommand=\DotAfterNumber]
\setuphead  [section]    [number=no]
\setuphead  [subsection] [number=no]


\setuplist  [chapter]    [width=2em]
\setuplist  [section]    [width=2em,margin=2em]
\setuplist  [subsection] [width=3em]

% O que entra no sumário
\setupcombinedlist[content]
                  [list={chapter,section},
                          alternative=c]   


\starttext


\completecontent




\chapter{Divisões}
%\title{Capítulo não numerado e não incluído no sumário}

\input knuth

\section{A numbered section}

\input knuth
\subsubject{An unnumbered subsection}

\input knuth

\subsection{A numbered subsection}
\input knuth


\subject{An unnumbered section}
\input knuth

\subsection{A numbered subsection}
\input knuth

\subsubject{An unnumbered subsection}
\input knuth


\chapter{Etiam arcu ullamcorper cras}
\input knuth

\chapter{ipsum maecenas est mollis}
\input knuth

\chapter{parturient sociosqu risus magna}
\input knuth

\chapter{convallis eros elit tempus}
\input knuth

\chapter{phasellus condimentum nibh}
\input knuth

\chapter{nullam accumsan consectetuer}
\input knuth

\chapter{diam gravida platea taciti}
\input knuth

\chapter{placerat scelerisque dapibus}
\input knuth

\chapter{potenti ad eget mus finibus}
\input knuth



\stoptext

