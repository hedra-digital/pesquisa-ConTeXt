\usemodule[tikz]
\setuppapersize[width=205mm,height=275mm]
\setuplayout[grid=yes, width=125mm, height=265mm, backspace=15mm, topspace=15mm]

\setuppagenumbering
  [alternative=doublesided, % Manchas espelhadas
   location=bottom,         % Localização dos números de página
   location={header,right}, % Localização do cabeçalho
   style=normal]            % Estilo dos números de página

\usemodule [setups]
\definefontfamily [din][sans] [DIN]
                  [tf=file:DINPro.ttf,
                   it=file:DINPro-Italic.ttf,
                   bf=file:DINPro-Bold.ttf,
                   bi=file:DINPro-BoldItalic.ttf]

\definefont[tiny][name:DINPro*default at 6pt]


% Definindo que a marginnote apareça sempre na margem do corte
\setupmargindata[inmargin]
  [style=\tiny, width=50mm, align=middle, location=outer]

\define[1]\mymargin
  {\inmargin{\framed[width=50mm, align=normal, frame=off]{#1}}}

% Define o espaçamento entrelinhas
%\setupinterlinespace[line=5pt]

% Ativa a indentação inicial e define o valor
\setupindenting[yes, 3ex]

% Define itemize personalizado com o item como []
\definesymbol[boxitem][{[\hskip .5em]}] 
\defineitemgroup [boxitemize]
\setupitemgroup  [boxitemize] [symbol=boxitem]


\setupbodyfont [din,sans,12pt] 

%\part{1º ANO -- FAMÍLIAS E SOCIEDADE}
\starttext


\title{Apresentação}

\subject{NESTE VOLUME}

A PROPOSTA É ABORDAR AS DIFERENTES CONFIGURAÇÕES DAS FAMÍLIAS, AS
RELAÇÕES SOCIAIS E AS PESSOAS IDOSAS NA SOCIEDADE, TRABALHANDO-SE COM A
TEMÁTICA DA DIVERSIDADE, DO PRECONCEITO E DAS FORMAS DE COMBATÊ-LO, BEM
COMO COM HÁBITOS DE VIDA SAUDÁVEL E CONVIVÊNCIA RESPEITOSA. POR MEIO DA
PROBLEMATIZAÇÃO DA REALIDADE E DA INTEGRAÇÃO DAS HABILIDADES E DAS
COMPETÊNCIAS CURRICULARES, BUSCA-SE CONTEXTUALIZAR OS DIFERENTES TEMAS,
DE MODO A POSSIBILITAR A CONSTRUÇÃO SIGNIFICATIVA DA APRENDIZAGEM.
CONHECIMENTOS E REFLEXÕES COMO ESSES TÊM O POTENCIAL DE CONTRIBUIR PARA
MUDANÇAS DE ATITUDE E DE LEVAR AS PESSOAS A PENSAREM E AGIREM DE FORMA
HUMANA, RESPEITOSA E MAIS POSITIVA.

\subject{MENSAGEM DA AUTORA}

%\textbf{\textless INSERIR FOTO DA ROSELAINE, A MESMA AUTORA DO TBNSI.\textgreater{}}

COMO PEDAGOGA, PSICOPEDAGOGA E MÃE, COLABORAR COM ESTA COLEÇÃO
SIGNIFICA, ALÉM DE UMA HONRA, UMA OPORTUNIDADE GRANDIOSA DE CONTRIBUIR
COM A AMPLIAÇÃO DE UM REPERTÓRIO FUNDAMENTAL PARA A FORMAÇÃO HUMANA E
INTEGRAL DE NOSSAS CRIANÇAS. EM SALA DE AULA, NA ASSESSORIA
INSTITUCIONAL, NO MESTRADO EM EDUCAÇÃO E SAÚDE E EM DIVERSOS PROJETOS
NOS QUAIS ATUEI, TIVE OPORTUNIDADE DE IDENTIFICAR NA PRÁTICA A URGÊNCIA
E A IMPORTÂNCIA DE TEMAS COMO ESSES SEREM ABORDADOS EM DIVERSOS
MOMENTOS, POR DIFERENTES ÁREAS DO CONHECIMENTO E DESDE A MAIS TENRA
IDADE. QUE TENHAMOS ÊXITO NA SUPERAÇÃO DE CONCEPÇÕES ULTRAPASSADAS E NA
CONSTRUÇÃO DE UM CONHECIMENTO MAIS SISTÊMICO, CONTEMPORÂNEO, DIVERSO E
COLETIVO.

\subject{ESTRUTURA DO LIVRO}

A COLEÇÃO TRAVESSIA ESTÁ DIVIDIDA EM 9 VOLUMES, E CADA VOLUME ESTÁ
VOLTADO PARA UM ANO ESPECÍFICO DO ENSINO FUNDAMENTAL, CONTENDO PELO
MENOS UM TEMA TRANSVERSAL CONTEMPORÂNEO, ENTRE OS TREZE PRINCIPAIS QUE
SÃO PROPOSTOS PELA BASE NACIONAL COMUM CURRICULAR (BNCC). CADA VOLUME
CONTA COM ESTES ELEMENTOS:

%\textbf{\textless Compor a dupla usando miniaturas das páginas com os elementos do projeto editorial, apresentando cada um com um título e um descritivo (como o que consta no arquivo do projeto).\textgreater{}}

\title{CAPÍTULO 1 -- AS FAMÍLIAS}

%\includegraphics[width=2.6842in,height=4.02609in]{media/image1.jpeg}
%\textbf{https://br.freepik.com/fotos-gratis/bebe-negro-passando-tempo-com-o-pai\_29011835.htm\#query=black\%20family\&position=5\&from\_view=search\&track=ais}

\startitemize[packed]
\item O QUE A IMAGEM DESTA PÁGINA REPRESENTA?

\item VOCÊ CONSIDERA QUE ESTA IMAGEM REPRESENTA UMA FAMÍLIA?

\item TODAS AS FAMÍLIAS SÃO IGUAIS?

\item ESSA FORMA DE FAMÍLIA SE PARECE COM A DA SUA?
\stopitemize

\subject{O QUE É UMA FAMÍLIA?}

\inmargin{{\bf(EF15LP09)} Expressar-se em situações de intercâmbio oral com
clareza, preocupando-se em ser compreendido pelo interlocutor e usando a
palavra com tom de voz audível, boa articulação e ritmo adequado.

{\bf(EF01HI01)} Identificar aspectos do seu crescimento por meio do
registro das lembranças particulares ou de lembranças dos membros de sua
família e/ou de sua comunidade.}

FAMÍLIA É UM GRUPO DE PESSOAS QUE POSSUI GRAU DE PARENTESCO OU LAÇO
AFETIVO. EM ALGUNS CASOS, ESSAS PESSOAS CONVIVEM EM UMA MESMA CASA. EM
OUTROS, NÃO.

TODA CRIANÇA TEM DIREITO A TER UMA FAMÍLIA PARA AMÁ-LA E PROTEGÊ-LA. NAS
FAMÍLIAS, É IMPORTANTE QUE EXISTA AMOR E RESPEITO ENTRE TODOS.

EXISTEM MUITOS TIPOS DE FAMÍLIA, E CADA UMA DELAS É DE UM JEITO. ELAS
PODEM SER GRANDES, COM MUITAS PESSOAS, OU PEQUENAS, COMO NA IMAGEM DE
ABERTURA DO CAPÍTULO, EM QUE VEMOS UM PAI SOLO COM SUA FILHA.

COMO É A SUA FAMÍLIA?

%\textbf{\textless abre seção Papel e caneta \textgreater{}}

NUMA FOLHA DE SULFITE, FAÇA UM DESENHO DA SUA FAMÍLIA. EM SEGUIDA,
MOSTRE-O PARA A TURMA E CONTE COMO ELA É E QUEM FAZ PARTE DELA.

%\textbf{\textless fecha seção Papel e caneta \textgreater{}}

\subject{A FAMÍLIA DA GENTE}

CADA FAMÍLIA TEM UMA HISTÓRIA E UM JEITO DE SE RELACIONAR. TODAS ELAS
TÊM MEMÓRIAS SOBRE MOMENTOS GOSTOSOS, RECEITAS FAVORITAS E VIAGENS QUE
DESFRUTARAM EM GRUPOS, POR EXEMPLO.

VOCÊ CONTOU SOBRE SUA FAMÍLIA E CONHECEU UM POUCO SOBRE A FAMÍLIA DE
SEUS COLEGAS. AGORA, IMAGINE QUE CADA FAMÍLIA É COMO SE FOSSE UMA
ÁRVORE. OS BISAVÓS, POR TEREM NASCIDO ANTES E SEREM DE GERAÇÕES MAIS
ANTIGAS, SÃO COMO AS RAÍZES DESSA ÁRVORE.

\inmargin{Peça aos alunos que mostrem suas produções e instigue-os a falarem sobre
suas famílias e dizerem quem faz parte delas. Ajude-os a perceber que as
famílias são diferentes e que são formadas por pessoas não apenas com
vínculos de sangue, mas também com vínculos de afeto.


%\textbf{\textless fecha boxe Palavra-chave\textgreater{}}

Relembre a atividade anterior realizada pelos alunos. Peça a ajuda deles
para se recordar do momento em que falaram sobre sua família e
conheceram as famílias dos colegas. Questione se as famílias que
conheceram eram todas iguais. Em seguida, peça que contem o que as
diferenciava. Esteja atento para favorecer a compreensão e o respeito
das crianças sobre os diferentes tipos de família. Reforce que cada uma
tem sua própria história e que todas as famílias merecem respeito. Esses
questionamentos valorizam a história de cada um e proporcionam uma
compreensão de identidade individual e coletiva. Aproveite para
perguntar se eles sabem quem é a pessoa mais velha da família e a mais
nova e os convide a fazer sua própria árvore genealógica.

%\textbf{\textless abre seção InvestigAÇÃO \textgreater{}}
}

OS AVÓS SÃO COMO O TRONCO: DELES VIERAM A MÃE, O PAI, AS TIAS OU OS
TIOS. NA REPRESENTAÇÃO DA ÁRVORE, ESSAS PESSOAS SERIAM OS GALHOS.


COM A UNIÃO ENTRE PESSOAS QUE SE CASAM OU PASSAM A VIVER JUNTAS, AS
ÁRVORES FAMILIARES TENDEM A CRESCER E OS GALHOS SE ENCHEM AINDA MAIS.

AS CRIANÇAS, NESSA ÁRVORE, SERIAM OS GRAVETOS, QUE UM DIA VÃO CRESCER E
SE TRANSFORMAR EM NOVOS GALHOS.

AS ÁRVORES FAMILIARES TAMBÉM PODEM SER CHAMADAS DE {\bf ÁRVORES GENEALÓGICAS}.

%\textbf{\textless Ricardo: ilustrar para este espaço um exemplo de árvore genealógica no esquema descrito no texto acima, com as gerações até os bisavós.\textgreater{}}

%\textbf{\textless abre boxe Palavra-chave\textgreater{}}

ÁRVORE GENEALÓGICA É UMA IMAGEM QUE REPRESENTA AS PESSOAS QUE TIVERAM
PARTICIPAÇÃO NA EXISTÊNCIA OU NA FAMÍLIA DE ALGUÉM.

ASSIM COMO CADA FAMÍLIA É DE UM JEITO, CADA ÁRVORE TAMBÉM É DIFERENTE DA
OUTRA. EXISTEM ÁRVORES COM POUCOS GALHOS, ASSIM COMO EXISTEM FAMÍLIAS
COM POUCOS INTEGRANTES. OUTRAS ÁRVORES PODEM SER MAIORES, ASSIM COMO AS
FAMÍLIAS QUE TÊM MUITAS PESSOAS.

FAÇA UMA PESQUISA JUNTO À SUA FAMÍLIA E, USANDO UMA CARTOLINA PARA FAZER
O ESQUEMA, CONSTRUA SUA PRÓPRIA ÁRVORE GENEALÓGICA. UMA SUGESTÃO
INTERESSANTE É COLAR FOTOS DO MAIOR NÚMERO POSSÍVEL DE PARENTES NOS
ESPAÇOS DEDICADOS A ESSAS PESSOAS NA ÁRVORE.

SERÁ QUE NA SUA TURMA HÁ FAMÍLIAS MUITO DIFERENTES? SIGA A ORIENTAÇÃO DO
PROFESSOR E REÚNA-SE EM GRUPO PARA CONHECER DIFERENTES ÁRVORES
GENEALÓGICAS E APRENDER MAIS SOBRE AS FAMÍLIAS DE SEUS COLEGAS.
APROVEITE PARA TAMBÉM COMPARTILHAR SUAS DESCOBERTAS SOBRE A SUA FAMÍLIA.

%\textbf{\textless fecha seção InvestigAÇÃO \textgreater{}}

\subject{POR QUE A FAMÍLIA É IMPORTANTE?}

A FAMÍLIA É COMPOSTA DAS PESSOAS COM QUEM VOCÊ TEM SUAS PRIMEIRAS
INTERAÇÕES E APRENDIZAGENS. A FAMÍLIA DEVE CUIDAR DE VOCÊ, ALIMENTAR
VOCÊ E MANTER VOCÊ SEGURO E PROTEGIDO.

AS PESSOAS DA FAMÍLIA SÃO QUEM GERALMENTE ENSINA OS HÁBITOS DE VIDA
DIÁRIA, COMO COMER E TOMAR BANHO, COMPORTAR-SE E TER BOAS MANEIRAS.
ESSAS PESSOAS SÃO IMPORTANTES PARA O CRESCIMENTO E PARA A FORMAÇÃO DAS
CRIANÇAS.

O QUE SUA FAMÍLIA ENSINOU A VOCÊ?

\subject{ATIVIDADES}

\inmargin{{\bf(EF02LP14)}~Planejar e produzir pequenos relatos de observação
de processos, de fatos, de experiências pessoais, mantendo as
características do gênero, considerando a situação comunicativa e o
tema/assunto do texto.

Construir a árvore genealógica favorece a compreensão da criança sobre
sua família e a ajuda a entender de onde ela vem. Conhecer nossa
linhagem e ter oportunidade de dialogar sobre nossos ancestrais é
importante para a construção de nossa identidade. Aproveite a proposta e
dialogue sobre esses pontos com os alunos. Como desdobramento da
atividade, proponha a formação de grupos para que as crianças falem
sobre suas árvores genealógicas e compartilhem suas descobertas com os
colegas. Incentive-os a mostrar a árvore e a contar sobre os integrantes
da família que aparecem nela. Pergunte quem ajudou a construir a árvore
e como foi esse momento.}

CONVERSE COM SUA TURMA E, COM A AJUDA DO PROFESSOR, RESPONDA ÀS QUESTÕES
A SEGUIR.

\startitemize[n]
\item O QUE É UMA FAMÍLIA?

\noindent\underbar{Família é um grupo de pessoas que possui grau de parentesco ou laço
afetivo.}

\item QUEM PODE FAZER PARTE DE UMA FAMÍLIA?

\noindent\underbar{A família pode ser constituída por pessoas com vínculo de parentesco ou
de laços afetivos. Relembre à turma que cada um tem sua família e sua
história e que, mesmo que a família do outro seja diferente, todas
merecem respeito.}

\item TODAS AS FAMÍLIAS SÃO IGUAIS? POR QUÊ?

\noindent\underbar{Cada família é única; por isso, todas são diferentes. Retome a atividade
da árvore genealógica e chame atenção para o fato de que o que
diferencia as famílias pode ser a quantidade de integrantes ou o fato de
conviverem na mesma casa ou em casas diferentes, além da forma como cada
uma delas vive e se organiza.}

\item A FAMÍLIA É IMPORTANTE? POR QUÊ?

\noindent\underbar{Ajude os alunos a perceber a importância da família para seu crescimento
e sua formação. Comente que a família é quem ensina as primeiras
aprendizagens, cuida, alimenta e mantém as crianças seguras e
protegidas.}

\item MARQUE COM UM X AQUILO QUE CORRESPONDE AO QUE VOCÊ JÁ APRENDEU COM
SUA FAMÍLIA.

\startboxitemize
\item ESCOVAR OS DENTES.

\item AJUDAR UM AMIGO.

\item FAZER SERVIÇOS DOMÉSTICOS, COMO ARRUMAR A CAMA OU TIRAR O PRATO DA MESA AO TERMINAR DE COMER.

\item SER GENTIL COM AS PESSOAS.

\item ARRUMAR SUA MOCHILA.

\item PREPARAR UM ALIMENTO.

\item TOMAR BANHO SEM AJUDA.

\item PEDIR AJUDA QUANDO NECESSÁRIO.
\stopboxitemize

\inmargin{Oriente a realização da atividade de forma coletiva. Leia o enunciado e
as alternativas com os alunos e peça que vão marcando aquelas que
correspondem às aprendizagens que tiveram com sua família. Conhecer o
que os alunos já sabem fazer sozinhos ajuda a compreender seu nível de
autonomia para a realização das atividades. Isso tem uma influência
importante na aprendizagem e pode trazer algumas ideias de como mediar
esse processo de forma cada vez mais adequada.}
\stopitemize