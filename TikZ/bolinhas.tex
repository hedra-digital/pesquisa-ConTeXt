\usemodule[tikz]
\usetikzlibrary[patterns,patterns.meta]
\starttext


\tikzdeclarepattern{
  name=mystars,
  type=uncolored,
  bounding box={(-5pt,-5pt) and (5pt,5pt)},
  tile size={(\tikztilesize,\tikztilesize)},
  parameters={\tikzstarpoints,\tikzstarradius,\tikzstarrotate,\tikztilesize},
  tile transformation={rotate=\tikzstarrotate},
  defaults={
    points/.store in=\tikzstarpoints,points=5,
    radius/.store in=\tikzstarradius,radius=3pt,
    rotate/.store in=\tikzstarrotate,rotate=0,
    tile size/.store in=\tikztilesize,tile size=10pt,
  },
  code={
    \pgfmathparse{180/\tikzstarpoints}\let\a=\pgfmathresult
    \fill (90:\tikzstarradius) \foreach \i in {1,...,\tikzstarpoints}{
      -- (90+2*\i*\a-\a:\tikzstarradius/2) -- (90+2*\i*\a:\tikzstarradius)
    } -- cycle;
  }
}

\starttikzpicture
% \draw[pattern=mystars,pattern color=blue]               (0,0) rectangle +(2,2);
% \draw[pattern={mystars[points=7,tile size=15pt]}]       (2,0) rectangle +(2,2);
% \draw[pattern={mystars[rotate=45]},pattern color=red]   (0,2) rectangle +(2,2);
 \fill[pattern color=blue,pattern={mystars[rotate=30,points=4,radius=5pt]}] (2,2) rectangle +(20,2);
\stoptikzpicture

\stoptext