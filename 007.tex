% To fit all the examples on one page, we need a taller page
\definepapersize[tall][width=15cm, height=42cm]
\setuppapersize[tall]

\defineparagraphs[whiting]
    [n=3,
    before={\blank[none]},     % Minimal padding, please.
    after={\blank[nowhite]}]   %

\definestartstop[whiteafter][
    before={\blank[none]} ,    % No padding, please
    after={\blank[medium]}     % The blank of interest
    ]
\definestartstop[whitebefore]
    [before={\blank[medium]},  % The blank of interest
    after={\blank[none]}       % No padding, please
    ]

\def\example#1{%
    \startwhiting
        \type{#1} \crlf
        Antelope, caribou, ocelot.
        \blank[#1]
        Don’t want an antelope nibbling the hoops.
    \whiting
        medium + {\tt #1}
        % No crlf here, or it'll combine with \blank[none] to create a blank line anyway.
        % (I assume it's equivalent to \crlf\crlf, or something.)
        \startwhiteafter
            Antelope, caribou, ocelot.
        \stopwhiteafter
        \blank[#1]
        Don’t want an antelope nibbling the hoops.
    \whiting
        {\tt #1} + medium \crlf
        Antelope, caribou, ocelot.
        \blank[#1]
        \startwhitebefore
            Don’t want an antelope nibbling the hoops.
        \stopwhitebefore
    \stopwhiting
    \hairline
}
\starttext

The table below has a row for each \type{\blank[#1]} keyword, and three columns:
\startitemize[packed]
    \item column 1 demonstrates \type{\blank[#1]} on its own;
    \item column 2 shows what \type{\blank[#1]} does when it comes after a \type{\blank[medium]};
    \item column 3 shows what \type{\blank[#1]} does when it comes before a \type{\blank[medium]}.
\stopitemize

\example{small}
\example{medium}
\example{big}
\example{nowhite}
\example{back}
\example{white}
\example{disable}
\example{line}
\example{halfline}
\example{formula}
\example{fixed}
\example{flexible}
\example{none}
\example{samepage}

\stoptext