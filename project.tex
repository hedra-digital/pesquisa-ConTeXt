% Coleção = SAEB
\startproject documents 

An example of a project file.

% Layout 
\environment layout

\product teacher
\product pupil
\product curriculum

\stopproject


..............
% teacher.tex

\startproduct teacher

\project documents

\component teacher1
\component teacher2

\stopproduct


..............
%teacher2
\startcomponent teacher2

\project documents
\product teacher

... text ...

\stopcomponent


======================================


% Coleção = SAEB
\startproject SAEB

An example of a project file.

% Layout 
\environment layout

\product aluno
\product professor
\product coordenador
\product cartazes

\stopproject


..............
% aluno.tex

\startproduct aluno

\project documents

\component 1M
\component 1P

\stopproduct


..............
% 1M.tex
\startcomponent 1M

\project SAEB
\product aluno

... text ...

\stopcomponent


======================


In principal a project file contains only a list of products and environments.
If you would process the project file all products will be placed in one
document. This is seldom wanted. This manual for example has a project
structure. Every part is a product and every chapter is a component. There are
several environments that are loaded in the main project file.

Project = Primeiro Ano
  > Product = Matemática
  	 > Component = Capítulos


===================================
environmemnt 


It is good practice to put all setups in one environment. In case a component or product has a
different layout you could define localenvironments:

\startlocalenvironment[names]
... setups ...
\stoplocalenvironment